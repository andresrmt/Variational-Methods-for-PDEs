\documentclass[a4paper,doc,11pt]{article}
%----------------------------------------------------------------------------------------
%	Paquetes y configuraciones
%----------------------------------------------------------------------------------------
\usepackage[numbers]{natbib}
\bibliographystyle{apalike}

\usepackage{amsfonts}
\usepackage{amsmath}
\usepackage{amssymb,amsthm}
\usepackage{enumerate}
\usepackage{enumitem}
\usepackage[utf8]{inputenc}
\usepackage[T1]{fontenc}
\usepackage{geometry}
\usepackage{hyperref}
\geometry{left=2cm,right=2cm,top=2.5cm,bottom=2.5cm}



\usepackage{url}
\def\UrlBreaks{\do\/\do-}
\usepackage{multirow}
\usepackage{multicol}
\usepackage{enumitem}
\usepackage{nicefrac}
\usepackage{graphicx}
\usepackage{stmaryrd}
\usepackage{dsfont}
\usepackage{bropd}
\usepackage{easybmat}
\usepackage{setspace}
\usepackage{comment}
\usepackage{mathpazo}
\usepackage{array}
\usepackage{commath}

\usepackage{sectsty}
\sectionfont{\centering\fontsize{13}{15}\selectfont}
\subsectionfont{\centering\fontsize{10}{10}\selectfont\scshape}

\newtheorem{theorem}{Theorem}[section]
\newtheorem{corollary}{Corollary}[theorem]
\newtheorem{proposition}{Proposition}[theorem]
\newtheorem{lemma}[theorem]{Lemma}
\newtheorem{definition}[theorem]{Definition}
\newtheorem{remark}[theorem]{Remark}
\newtheorem{example}[theorem]{Example}
\newtheorem{claim}{Claim}[subsection]




\usepackage[font=small]{caption}
\usepackage[font=small]{subcaption}
\captionsetup{subrefformat=parens}
\usepackage{booktabs} % nice headers for tables

\newcommand{\R}{\mathbb{R}}
\newcommand{\Z}{\mathbb{Z}}
\newcommand{\N}{\mathbb{N}}
\newcommand{\CC}{\mathcal{C}}
\newcommand{\llb}{\llbracket}
\newcommand{\rrb}{\rrbracket}

\DeclareMathOperator{\dom}{dom}
\DeclareMathOperator{\supp}{supp}
\setcounter{MaxMatrixCols}{20}


\SetLabelAlign{parright}{\parbox[t]{\labelwidth}{\raggedright#1}}
\allowdisplaybreaks


\usepackage[symbol]{footmisc}

\renewcommand{\thefootnote}{\fnsymbol{footnote}}


\linespread{1.38}

%---------------------------------------- Autoría ---------------------------------------- %
\usepackage{titling}
\predate{}
\postdate{\vspace{-2\baselineskip}}



\title{\bf
    \Large
    SMSTC 2021 
    \\
    Variational Methods of PDEs
}
\author{}%Andrés Miniguano Trujillo}
\date{}


\begin{document}
%\pagenumbering{Roman} 
\maketitle




%\newpage
%\tableofcontents


%%%%%%%%%%%%%%%%%%%%%%%%%%%%%%%%%%%%%%%%%%%%%%%%%%%
%\newpage
\section{User's Guide to Sobolev Spaces}
%\pagenumbering{arabic}

%%%%%%%%%%%%%%%%%
%%%%%%%%%%%%%%%%%
\subsection{Basic definitions}

Let \(\Omega \subset \R^N\) be an open set, \(N\) a positive integer, and \(p \in \R\) be such that \( 1\leq p \leq \infty\).

If we recall from Multi-variable Calculus, we know that if \(u\) is of class \( \CC^1(\Omega)\), then integration by parts yields
\[
    \int\limits_\Omega u \pd{\varphi}{x_i} \dif x = -\int\limits_\Omega \pd{u}{x_i} \varphi \dif x,
\]
whenever \( \varphi \equiv 0\) on \(\partial \Omega\). This gives us an alternative definition for the derivative of \(u\), which we can extend for measurable functions: Let \(u \in L^p(\Omega)\) and \(i \in \{1,\ldots,N\}\). If there exists \(g_i \in L^p (\Omega)\) such that
\begin{equation}
    \label{eq:1-pd}
    \int\limits_\Omega u \pd{\varphi}{x_i} \dif x = -\int\limits_\Omega g_i \varphi \dif x,
\end{equation}
for all \(\varphi \in \CC_c^{1} (\Omega)\)\footnote[2]{We denote \(\CC_c^{\infty} (\Omega)\) as the set of compactly supported functions of class \(\CC^1\) defined over \(\Omega\): 
\[
    \CC_c^{1} (\Omega) = \big\{ \varphi \in \CC^1(\Omega): \, \Omega \supset \supp \varphi = \{x: \varphi(x) \neq 0\} \text{ is compact } \big\}.
\]
}, then we say that \(g_i\) is the \emph{weak partial derivative of \(u\) with respect to \(x_i\)}, and for convenience we denote \( \pd{u}{x_i} := g_i \). Likewise, if there exist \( g_1, \ldots g_N \in L^p (\Omega)\) such that \eqref{eq:1-pd} is satisfied for all \( i \in \{1, \ldots, N\}\), then we say that the vector
\(
    \nabla u = 
    \begin{pmatrix}
        \pd{u}{x_1}, \ldots, \pd{u}{x_N}
    \end{pmatrix}
\)
is the \emph{weak derivative of \(u\)}.

The set of \(L^p (\Omega)\) functions that have weak derivatives form the \emph{Sobolev space} \( W^{1,p} (\Omega)\) defined by
\[
    W^{1,p} (\Omega) :=
    \left\{
        u \in L^p (\Omega): \,
        \begin{aligned}
        &\exists \{g_i\}_{i\in \{1,\ldots, N\}} \text{ such that } 
        \\
        &\int\limits_\Omega u \pd{\varphi}{x_i} \dif x = -\int\limits_\Omega g_i \varphi \dif x \quad \forall \varphi \in \CC_c^1 (\Omega) \quad \forall i \in \{1,\ldots,N\}
        \end{aligned}
    \right\}.
\]
This space is equipped with the norm\footnote{When there is no confusion, we will write \(W^{1,p}\) instead of \(W^{1,p} (\Omega)\).}
\[
    \|u\|_{W^{1,p}} :=  \|u\|_p + \sum_{i=1}^N \left\| \pd{u}{x_i} \right\|_p.
\]
We further note \(H^1(\Omega) := W^{1,2}(\Omega)\), which has the associated scalar product
\[
    (u,v)_{H^1} = (u,v)_{L^2} + \sum_{i=1}^N \left( \pd{u}{x_i} , \pd{v}{x_i} \right)_{L^2}.
\]
Notice that by the Cauchy–Bunyakovsky–Schwarz inequality the associated norm in  \(H^1\) is equivalent to the \(W^{1,2}\) norm.

\begin{proposition}
    The weak derivative is unique (almost everywhere).
\end{proposition}
\begin{proof}
    Let \(u \in W^{1,p}(\Omega)\), and suppose that there exists \(g_i\) and \( h_i\) such that both satisfy equation \eqref{eq:1-pd}, then
    \[
        0 = \int\limits_\Omega u \pd{\varphi}{x_i} \dif x - \int\limits_\Omega u \pd{\varphi}{x_i} \dif x = \int\limits_\Omega (g_i - h_i) \varphi \dif x
    \]
    for all \(\varphi \in \CC_c^1 (\Omega)\). By the Fundamental lemma of the calculus of variations, we have that \( g_i - h_i = 0\) almost everywhere, that is \( g_i = h_i\) (a.e).
\end{proof}

As weak derivatives are unique, there is no ambiguity in writing \(g_i = \pd{u}{x_i}\); i.e., our notation is consistent. Finally, we will see an important example of this theory:

\begin{example}
    Let \(\Omega = (-1,1) \subset \R\) and consider the function \( u(x) = |x|\). We have that integration by parts yields
    \begin{align*}
        \int\limits_\Omega u \varphi' \dif x
        =
        \int\limits_{0}^1 x \varphi'  \dif x
        -
        \int\limits_{-1}^0 x \varphi'  \dif x
        = 
        -\int\limits_{0}^1 \varphi  \dif x
        +
        \int\limits_{-1}^0 \varphi  \dif x
        =
        -\int\limits_\Omega g \varphi \dif x,
    \end{align*}
    where \(g\) is defined as
    \[
        g(x) = 
        \begin{cases}
            +1 & \text{if } x\in (0,1),
            \\
            -1 & \text{if } x\in (-1,0).
        \end{cases}
    \]
    Notice that by definition \(u\in L^p(\Omega)\) for every \(1 \leq p\leq \infty \), and it is clear that \(g \in L^p (\Omega)\) as well. 
\end{example}





%%%%%%%%%%%%%%%%%
%%%%%%%%%%%%%%%%%


\vspace{2\baselineskip}

Example cite:
\citet{Yann1991}, \citep{Yann1991}





%%%%%%%%%%%%%%%%%%%%%%%%%%%%%%%%%%%%%%%%%


\section*{Availability of data, material, and code}
{
%\small

All the files and this document are available as in the following repository:
\begin{quote}
    \noindent \href{https://github.com/andresrmt/Variational-Methods-for-PDEs}{\texttt{https://github.com/andresrmt/Variational-Methods-for-PDEs}}
\end{quote}



}

%%%%%%%%%%%%%%%%%%%%%%%%%%%%%%%%%%%%%%%%%
%%%%%%%%%%%%%%%%%%%%%%%%%%%%%%%%%%%%%%%%%
\newpage


\bibliography{Cites}

\end{document}



